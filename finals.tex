\documentclass{article}
\usepackage{amsmath}
\usepackage{amssymb}
\usepackage{geometry}
\usepackage[linguistics]{forest}
\usepackage{microtype}
\usepackage{float}
\usepackage{makecell}
\usepackage{hyperref}
\usepackage{hypcap}
\usepackage{multirow}
\usepackage{colortbl}
\usepackage{hhline}

\geometry{legalpaper, portrait, margin=1in}
\title{Phil 120, Finals}
\author{Ray Gong}
\date{June 25, 2021}

\newcommand{\true}{{\mathfrak{t}}}
\newcommand{\false}{{\mathfrak{f}}}
\newcommand{\greencell}{\cellcolor{green!45}T}
\newcommand{\redcell}{\cellcolor{red!45}F}
\newcommand{\orangecell}{\cellcolor{orange!45}B}
\newcommand{\E}{\mathbb{E}}

\begin{document}
\maketitle
\newpage
\Large

\section*{Q1}
\begin{tabular}{c c c | c c c c c c c}
    p & q & r & $(\lnot p$ & $\wedge$ & $(q \vee$ & $\lnot r ))$ & $\leftrightarrow$ & $(\lnot q$ & $\rightarrow p)$\\
    \hline
    F & F & F & T & T & T& T & F & T & F\\
    F & F & T & T & F & F& F & T & T & F\\
    F & T & F & T & T & T& T & T & F & T\\
    F & T & T & T & T & T& F & T & F & T\\
    T & F & F & F & F & T& T & F & T & T\\
    T & F & T & F & F & F& F & F & T & T\\
    T & T & F & F & F & T& T & T & F & T\\
    T & T & T & F & F & T& F & T & F & T\\
\end{tabular}

so, the formula is not a) a logical validity, b) is contingent, c) is not a falsehood.


\section*{Q2}
\subsection*{a)}
let a be alfred, k be kurt, Lx if x proves a lemma, Tx if x proves a theorem, Px if x writes a paper.\\
$(Tk \vee Lk) \rightarrow Pa$

\subsection*{b)}
let a be alfred, b be albert, Cx if x is cooking, Rx if it is raining where x is, Sx if x is shopping.\\
$\lnot Sa \wedge (\lnot Rb \rightarrow Cb)$

\subsection*{c)}
let s be snow, let t be 2 + 2, f be 5, u be tuesday,  Wx if x is white, Tx if today's weekday is x.\\
$Ws \rightarrow (t = f \vee \lnot Tu)$
\subsection*{d)}

let a be alfred, k be kurt, Hx if x is on a hike, Cxy if x is chatting with y.\\
$Ha \rightarrow \lnot Cka$

\newpage
\section*{Q3}
\begin{forest}
    [$\true @w \exists x \Diamond Ax$
        [$\false @w \Diamond \exists x \ Ax$
            [$\true @w \E b$
                [$\true @w \diamond Ab$
                    [$\true @w_1 Ab $
                        [$\false @w_1 \exists x Ax$
                            [$\false @w_1 \E b$\\$\times_1$]
                            [$\false @w_1 Ab$\\$\times_1$]
                        ]
                    ]
                ]
            ]
        ]
    ]
\end{forest}

since all branches closed with rule 1, the statement holds in CL, k3, LP, and fde.

\section*{Q4}

suppose there are 2 worlds in the universe, w1 and w2 such that C is true in w1, but false in w2. \\
since $\lnot \square C \iff \diamond \lnot C$(proof is trivial from tableaux), 
$\square \lnot \square C \iff \square \diamond \lnot C$. 
Then since $\lnot C$ is satisfied in at least world, $\diamond \lnot C$ holds in every worlds, which means
 $\square \diamond \lnot$ holds too. However, $\lnot C$ does not hold in every world, so $\square \lnot C$ is false,
 and so this is a counter example for the implication statement($\rightarrow$).

\section*{Q5}

\subsection*{a)}

Suppose  $A \subseteq B$, then for any $(a_1, a_2, ... a_i) \in \mathcal{P}(A), a_1, a_2...a_i \in A \subseteq B $, then 
$a_1, a_2...a_i \in B \implies (a_1, a_2...a_i) \in \mathcal{P}(B)$ by definition of powerset, and so a) holds.

\subsection*{b)}
$|C \times C \times C| = |C|^3 = 27, 25 < 27 < 30$, so yes, the statement holds.


\section*{Q6}

\begin{forest}
    [$\true A \wedge (B \vee C)$
        [$\true B \rightarrow D$
            [$\true C \rightarrow \lnot E$
                [$\false E \rightarrow D$
                    [$\true A$
                        [$\true B \vee C$
                            [$\false \lnot E$
                                [$\false D $
                                    [$\true \lnot B$
                                        [$\true B$\\$\times$]
                                        [$\true C$
                                            [$\true \lnot C$\\$\times$]
                                            [$\true \lnot E$\\$\times$]
                                        ]
                                    ]
                                    [$\true D$\\$\times$]
                                ]
                            ]
                        ]
                    ]
                ]
            ]
        ]
    ]
\end{forest}\\
All branches closes, the consequence relation holds in TV.


\section*{Q7}

\subsection*{a)}
let Cx if x is car, Sx if x is small, Fx if x is fast.\\
$\exists x (Cx \wedge Sx \wedge Fx)$

\subsection*{b)}

let $\mathbb{N}, \mathbb{R}-\mathbb{Q}, \mathbb{R}$ be the set of natural, irrational, and real numbers. let $x\in A$ be the preicate that x is in the set A.
$x < y$ if x is less than y.\\

$\forall n \in \mathbb{N} (((2n +1) \in \mathbb{R}-\mathbb{Q}) \vee (\exists x \ (x \in \mathbb{R}  \wedge (2n + 1) < x)))$

\subsection*{c)}
let Sx if x is student, Bx if x is bike, Oxy if x owns y, Rxy if x rides y daily.\\
$\forall x (Sx \rightarrow (\exists y (By \wedge Oxy) \rightarrow Rxy))$


\section*{Q8}
\begin{forest}
    [$\true \forall (Px \rightarrow (Qx \vee \lnot Rx))$
        [$\true \forall x((Qx \wedge Rx) \rightarrow \lnot Px)$
            [$\true \exists x(Qx \wedge Rx)$
                [$\false \forall ((Px \wedge Qx) \rightarrow (Px \wedge \lnot Rx))$
                    [$\true Qa \wedge Ra$
                        [$\true Qa$
                            [$\true Ra$
                                [$\false (Pb \wedge Qb) \rightarrow (Pb \wedge \lnot Rb)$
                                    [$\true (Qa \wedge Ra) \rightarrow \lnot Pa$
                                        [$\true Pa \rightarrow (Qa \vee \lnot Ra)$
                                            [$\true \lnot Pa$
                                                [$\true \lnot(Qa \wedge Ra)$
                                                    [$\true \lnot Qa$\\$\times$]
                                                    [$\true \lnot Ra$\\$\times$]
                                                ]
                                                [$\true \lnot Pa$]
                                            ]
                                            [$\true Qa \vee \lnot Ra$
                                                [$\true Qa$]
                                                [$\true \lnot Ra$\\$\times$]
                                            ]
                                        ]
                                    ]
                                ]
                            ]
                        ]
                    ]
                ]
            ]
        ]
    ]
\end{forest}\\

looks like i choses the wrong variable names, but i dont have time to redo this question. most likely the first 2 
`forall' statements should take the same variable name as the name chosen for the conclusion.   


\section*{Q9}
let D = \{o1, o2\}, E = \{o1\}, $\delta(a) = o1, \delta(b) = o2$\\
+(F) = \{o1\}, -(F) = \{o2\}, +(G) = \{o1, o2\}, -(G) = $\emptyset$.\\

So, for all x, x = a since x can only be assigned o1, which denotes a. Fa is true so $Fa 
vee \lnot Gb$ is true, $\forall x (Fx \rightarrow Gx)$ holds since x can only be o1, and o1 is in the 
extension set of both F and G. however the conclusion is false since on existing object to be assigned to $y$ satisfy both 
$\lnot Fy$ and $\lnot Gy$

\section*{Q10}


\begin{tabular}{c c | c c c c c c c}
    p & q & $(\lnot p)$ & $\leftrightarrow$ & $\lnot$ & $(q \vee p)$ & $\rightarrow$ & $(p \wedge$ & $\lnot q$\\
    \hline
    F & F & T &       T & T & F & F & F & T\\
    F & T & T &       F & F & T & T & f & F\\
    F & N & T &       N & N & N & N & F & N\\
    T & F & F &       T & F & T & T & T & T\\
    T & T & F &       T & F & T & F & F & F\\
    T & N & F &       T & F & T & N & N & N\\
    N & F & N &       N & N & N & N & N & T\\
    N & T & N &       N & F & T & N & F & F\\
    N & N & N &       N & N & N & N & N & N\\

    
\end{tabular}


\section*{Q11}

\begin{forest}
    [$\true \lnot (A \wedge \lnot C) \wedge B$
        [$\false (\lnot A \vee C) \vee D$
            [$\false D$
                [$\false \lnot A$
                    [$\false C$
                        [$\true B$
                            [$\true \lnot A$\\$\times_1$]
                            [$\true \lnot \lnot C$
                                [$\true C$ \\$\times_1$]
                            ]
                        ]
                    ]
                ]
            ]
        ]
    ]
\end{forest}
all branches closed under rule 1, relation holds in fde.

\section*{Q12}
a)false. since there are no theorems in K3, its not possible to find a formula that is a theorem in both K3 and LP.\\


b) true. the shared content is that if the premise is satisfied(true or both), then the conclusion be be satisfied also.\\

c) true. truth table and tableaux both provide ways to determine truth value of a formula in all 4 logic theories.\\

d) true.

e)false. if a predicate is that `at most five of some thing', then in particular, it could be 0 or fewer, 
so `at least one...' is not implied here.
\end{document}