\documentclass{article}
\usepackage{amsmath}
\usepackage{amssymb}
\usepackage{geometry}
\usepackage[linguistics]{forest}

\usepackage{float}
\usepackage{makecell}
\newcommand{\true}{{\mathfrak{t}}}
\newcommand{\false}{{\mathfrak{f}}}
\geometry{legalpaper, portrait, margin=1in}
\title{Phil 120, Review Notes}
\author{Stuff}
\date{May 23, 2021}

\begin{document}
\maketitle
\newpage
\tableofcontents{}

\newpage    


\section{References}

\subsection{BasicLogic Operators}
The followings are for A * B, where '*' is an operator, A is top row, B is left column.


\begin{table}[H]
    \centering
    \large
    \begin{tabular}{c l}
        \begin{tabular}{c|c|c}
            $\wedge$ & T & F\\
            \hline
            T & T & F\\
            F & F & F\\
        \end{tabular} & {AND. Conjuction. A $\wedge$ B is true only when both A and B are true.}\\
        &\\
        \hline
        &\\
        \begin{tabular}{c|c|c}
            $\vee$ & T & F\\
            \hline
            T & T & T\\
            F & T & F
        \end{tabular} & OR. Disjunction. A $\vee$ B is true when either A or B, or Both are true. \\
        &\\
        \hline
        &\\
        \begin{tabular}{c|c|c}
            $\rightarrow$ & T & F\\
            \hline
            T & T & T\\
            F & F & T\\
        \end{tabular} & \makecell[l]{IMPLIES. If A then B. A implies B.\\
        A implies B is true when A is true and B is true, or when A is false.\\
        Note: $A \rightarrow B = \lnot A \vee B$}\\
        &\\
        \hline
        &\\
        \begin{tabular}{c|c|c}
            $\leftrightarrow$&  T & F\\
            \hline
            T & T & F\\
            F & F & T
        \end{tabular} & \makecell[l]{IFF, A if and only B. A is logically equivalent, two way implication.\\
        A $\leftrightarrow$ B is true exactly when the truth value of A is the same as B.\\
        Note: $A \leftrightarrow B = (A \rightarrow B) \wedge (B \rightarrow A) = (A \wedge B) \vee (\lnot A \wedge \lnot B)$
        }
    \end{tabular}
\end{table}


\subsection{Some Logic Identities}

\begin{table}[H]
    \centering
    \large
    \begin{tabular}{c l}
        $A \vee \lnot A = T$ & Excluded Middle, either A or not A must be true.\\
        &\\
        \hline
        &\\
        $\lnot (A \wedge \lnot A)$ & \makecell[l]{
            Non-contradiction.\\ It is true that not both A and not A hold at the \\same time. 
        }\\
        &\\
        \hline
        &\\
        $A \rightarrow B, A \implies B$ & \makecell[l]{Modus ponenes, to prove. \\
        If A implies and B and A is true, then B is true.}\\
        &\\
        \hline
        &\\
        $A \rightarrow B, \lnot B \implies \lnot A$ & \makecell[l]{Modus tollens, to disprove. \\
        If the conclusion is false, then the premise is false also.}\\
        &\\
        \hline
        &\\
        $A \vee B, \lnot A \implies B$ & \makecell[l]{
            Disjunctive syllogism.\\
            If at least one of A or B is true, then if one of them is \\false, the other must be true.
        }\\
        &\\
        \hline
        &\\
        $(A \rightarrow B) \iff (\lnot B \rightarrow \lnot A)$ & Contrapositive. Similar to Modus tollens.\\
        &\\
        \hline
        &\\
        $A, \lnot A \implies B$ & \makecell[l]{
            Explosion. \\
            From a false premise you can arrive at any conclusion.
        }\\
        &\\
        \hline
        &\\
        \makecell{
            $\lnot (A \vee B) \iff \lnot A \wedge \lnot B$\\
            $\lnot (A \wedge B) \iff \lnot A \vee \lnot B$
        } & De Morgan's Law.\\
        &\\
        \hline
        &\\
        \makecell{
            $A \vee (B \wedge C) \iff (A\vee B) \wedge (A \vee C)$\\
            $A \wedge (B \vee C) \iff (A \wedge B)\vee (A \wedge C)$
        } & Distributability

    \end{tabular}
\end{table} 

\subsection{Tableaux Identities}
\begin{table}[H]
    \centering
    \Large
    \begin{tabular}{c c | c c}
        $\true \wedge$: & 
        \begin{forest}
        [$\true A \wedge B$ 
            [$\true A$ \\ $\true B$]
        ]
        \end{forest}
        & $\false \wedge$: & 
        \begin{forest}
        [$\false A \wedge B$ 
            [$\false A$]  [$\false B$]
        ]
        \end{forest}\\
        
        %%%%%%%%%%%%%%%%%%%%%%%%%%%%%%%%%%%%%%%

        \hline
        $\true \vee$: & 
        \begin{forest}
            [$\true A \vee B$ 
                [$\true A$]  [$\true B$]
            ]
        \end{forest} 
        & $\false \vee$: 
        & \begin{forest}
            [$\false A \vee B$ 
                [$\false A$ \\ $\false B$]
            ]
            \end{forest}\\
        %%%%%%%%%%%%%%%%%%%%%%%%%%%%%%%%%%%%%%%
        \hline 
        $\true \rightarrow$: &
        \begin{forest}
            [$\true A \rightarrow B$
                [$\true \lnot A$]
                [$\true B $]
            ]
        \end{forest}
        & $\false  \rightarrow $: &
        \begin{forest}
            [$\false A \rightarrow B$ 
                [$\false \lnot A$ \\ $\false B$]
            ]
            \end{forest}\\
        %%%%%%%%%%%%%%%%%%%%%%%%%%%%%%%%%%%%%%%
        \hline
        $\true \leftrightarrow$ :
        & \begin{forest}
            [ $\true A \leftrightarrow B$
                [$\true \lnot A$ [$\true \lnot B$]]
                [$\true \lnot A$ [$\true A$]]
                [$\true \lnot B$ [$\true  B$]]
                [$\true  A$ [$\true  B$]]
            ]
        \end{forest}
        & $\false \leftrightarrow$: 
        & \begin{forest}
            [$\false A \leftrightarrow B$
            [$\false \lnot A$ [$\false B$]]
            [$\false \lnot B$ [$\false A$]]
            ]
        \end{forest}
    \end{tabular}
\end{table}
{\large Note: \\
Use De Morgan's Law to deal with negations($\lnot$), also, $\lnot (A \rightarrow B) = (\lnot B \wedge A)$

}
\end{document}