
\documentclass{article}

\usepackage{hhline}
\usepackage{xcolor}
\usepackage{colortbl}
\usepackage{float}
\usepackage{multirow}
\usepackage{hhline}
\newcommand{\RowColor}{\rowcolor{red!50} \cellcolor{white}}

\begin{document}
\begin{tabular}{l || l l l l l | l}
                  & 1 & 2 & 3 & 4 & 5 & \cellcolor{yellow}6 \\\hhline{=#=====|=}
\RowColor       1 & 1 & 1 & 1 & 1 & 1 & \cellcolor{white}0  \\
\RowColor       2 & 1 & 1 & 1 & 1 & 1 & \cellcolor{white}0  \\
\RowColor       3 & 1 & 1 & 1 & 1 & 1 & \cellcolor{white}0  \\
\RowColor       4 & 1 & 1 & 1 & 1 & 1 & \cellcolor{white}0  \\
\RowColor       5 & 1 & 1 & 1 & 1 & 1 & \cellcolor{white}0  \\\hline
\cellcolor{cyan}6 & 0 & 0 & 0 & 0 & 0 & \cellcolor{green}1
\end{tabular}



    \begin{tabular}{c l}

        
        \setlength\arrayrulewidth{1pt}
        \begin{tabular}{|l |c| l| c| c|}
            \RowColor \cellcolor{green}T & F & \cellcolor{green} U\\
            \hline
        \end{tabular}
    \end{tabular}

\end{document}
