\documentclass{article}
\usepackage{amsmath}
\usepackage{amssymb}
\usepackage{geometry}
\usepackage[linguistics]{forest}
\usepackage{microtype}
\usepackage{float}
\usepackage{makecell}
\newcommand{\true}{{\mathfrak{t}}}
\newcommand{\false}{{\mathfrak{f}}}
\geometry{legalpaper, portrait, margin=1in}
\title{Phil 120, Review Notes}
\author{Stuff}
\date{May 23, 2021}

\begin{document}
\maketitle
\newpage
\tableofcontents{}

\newpage    


\section{References}

\subsection{Basic Logic Operators}
The followings are for A * B, where '*' is an operator, A is top row, B is left column.


\begin{table}[H]
    \centering
    \large
    \begin{tabular}{c l}
        \begin{tabular}{c|c|c}
            $\wedge$ & T & F\\
            \hline
            T & T & F\\
            F & F & F\\
        \end{tabular} & {AND. Conjuction. A $\wedge$ B is true only when both A and B are true.}\\
        &\\
        \hline
        &\\
        \begin{tabular}{c|c|c}
            $\vee$ & T & F\\
            \hline
            T & T & T\\
            F & T & F
        \end{tabular} & OR. Disjunction. A $\vee$ B is true when either A or B, or Both are true. \\
        &\\
        \hline
        &\\
        \begin{tabular}{c|c|c}
            $\rightarrow$ & T & F\\
            \hline
            T & T & T\\
            F & F & T\\
        \end{tabular} & \makecell[l]{IMPLIES. If A then B. A implies B.\\
        A implies B is true when A is true and B is true, or when A is false.\\
        Note: $A \rightarrow B = \lnot A \vee B$}\\
        &\\
        \hline
        &\\
        \begin{tabular}{c|c|c}
            $\leftrightarrow$&  T & F\\
            \hline
            T & T & F\\
            F & F & T
        \end{tabular} & \makecell[l]{IFF, A if and only B. A is logically equivalent, two way implication.\\
        A $\leftrightarrow$ B is true exactly when the truth value of A is the same as B.\\
        Note: $A \leftrightarrow B = (A \rightarrow B) \wedge (B \rightarrow A) = (A \wedge B) \vee (\lnot A \wedge \lnot B)$
        }
    \end{tabular}
\end{table}


\subsection{Some Logic Identities}

\begin{table}[H]
    \centering
    \large
    \begin{tabular}{c l}
        $A \vee \lnot A = T$ & Excluded Middle, either A or not A must be true.\\
        &\\
        \hline
        &\\
        $\lnot (A \wedge \lnot A)$ & \makecell[l]{
            Non-contradiction.\\ It is true that not both A and not A hold at the \\same time. 
        }\\
        &\\
        \hline
        &\\
        $A \rightarrow B, A \implies B$ & \makecell[l]{Modus ponenes, to prove. \\
        If A implies and B and A is true, then B is true.}\\
        &\\
        \hline
        &\\
        $A \rightarrow B, \lnot B \implies \lnot A$ & \makecell[l]{Modus tollens, to disprove. \\
        If the conclusion is false, then the premise is false also.}\\
        &\\
        \hline
        &\\
        $A \vee B, \lnot A \implies B$ & \makecell[l]{
            Disjunctive syllogism.\\
            If at least one of A or B is true, then if one of them is \\false, the other must be true.
        }\\
        &\\
        \hline
        &\\
        $(A \rightarrow B) \iff (\lnot B \rightarrow \lnot A)$ & Contrapositive. Similar to Modus tollens.\\
        &\\
        \hline
        &\\
        $A, \lnot A \implies B$ & \makecell[l]{
            Explosion. \\
            From a false premise you can arrive at any conclusion.
        }\\
        &\\
        \hline
        &\\
        \makecell{
            $\lnot (A \vee B) \iff \lnot A \wedge \lnot B$\\
            $\lnot (A \wedge B) \iff \lnot A \vee \lnot B$
        } & De Morgan's Law.\\
        &\\
        \hline
        &\\
        \makecell{
            $A \vee (B \wedge C) \iff (A\vee B) \wedge (A \vee C)$\\
            $A \wedge (B \vee C) \iff (A \wedge B)\vee (A \wedge C)$
        } & Distributability

    \end{tabular}
\end{table} 

\subsection{Tableaux Identities}
\begin{table}[H]
    \centering
    \Large
    \begin{tabular}{c c | c c}
        $\true \wedge$: & 
        \begin{forest}
        [$\true A \wedge B$ 
            [$\true A$ \\ $\true B$]
        ]
        \end{forest}
        & $\false \wedge$: & 
        \begin{forest}
        [$\false A \wedge B$ 
            [$\false A$]  [$\false B$]
        ]
        \end{forest}\\
        
        %%%%%%%%%%%%%%%%%%%%%%%%%%%%%%%%%%%%%%%

        \hline
        $\true \vee$: & 
        \begin{forest}
            [$\true A \vee B$ 
                [$\true A$]  [$\true B$]
            ]
        \end{forest} 
        & $\false \vee$: 
        & \begin{forest}
            [$\false A \vee B$ 
                [$\false A$ \\ $\false B$]
            ]
            \end{forest}\\
        %%%%%%%%%%%%%%%%%%%%%%%%%%%%%%%%%%%%%%%
        \hline 
        $\true \rightarrow$: &
        \begin{forest}
            [$\true A \rightarrow B$
                [$\true \lnot A$]
                [$\true B $]
            ]
        \end{forest}
        & $\false  \rightarrow $: &
        \begin{forest}
            [$\false A \rightarrow B$ 
                [$\false \lnot A$ \\ $\false B$]
            ]
            \end{forest}\\
        %%%%%%%%%%%%%%%%%%%%%%%%%%%%%%%%%%%%%%%
        \hline
        $\true \leftrightarrow$ :
        & \begin{forest}
            [ $\true A \leftrightarrow B$
                [$\true \lnot A$ [$\true \lnot B$]]
                [$\true \lnot A$ [$\true A$]]
                [$\true \lnot B$ [$\true  B$]]
                [$\true  A$ [$\true  B$]]
            ]
        \end{forest}
        & $\false \leftrightarrow$: 
        & \begin{forest}
            [$\false A \leftrightarrow B$
            [$\false \lnot A$ [$\false B$]]
            [$\false \lnot B$ [$\false A$]]
            ]
        \end{forest}
    \end{tabular}
\end{table}
{\large Note: 
\begin{itemize}
    \item Use De Morgan's Law to deal with negations($\lnot$), also, $\lnot (A \rightarrow B) = (\lnot B \wedge A)$.
    \item To prove a consequence, set the premise to true, conclusion to false. This proves that there is
     no possible counter example, since the negation is never satisfied. 
     \begin{itemize}
         \item If there are atomic branches left open, then those are valid counter examples.
     \end{itemize}
    \item A branch is closed any of following pairs occurs in a branch:
    $\{(\false A, \true A), (\false A, \false \lnot A), (\true A, \true \lnot A)\}$, use a '$\times$' to 
    indicate a closed branch.
\end{itemize}
Example: Use tableaux to prove $ (E \rightarrow D) \vdash_1 ((D \wedge E) \leftrightarrow E)$\\


\begin{center}
    \begin{forest}
        [
            $\true (E \rightarrow D)$
            [$\false ((D\wedge E) \leftrightarrow E)$
                [$\true \lnot E$
                    [$\false \lnot (D \wedge E)$ 
                        [$\false E$
                            [$\false \lnot D$ 
                            [$\false \lnot E$\\$\times$]]
                        ]
                    ]
                    [$\false (D \wedge E)$ 
                        [$\false \lnot E$\\$\times $
                        ]
                    ]
                ]
                [$\true D$
                [$\false \lnot (D \wedge E)$ 
                    [$\false E$
                    [$\false \lnot D$ 
                    [$\false \lnot E$\\$\times$]]
                    ]
                ]
                [$\false (D \wedge E)$ 
                    [$\false \lnot E$
                    [$\false D$\\$\times$]
                    [$\false E$\\$\times$]
                    ]
                ]
                ]
            ]
        ]
        \end{forest}
\end{center}


Since all branches are closed, the negation of the conclusion is never satisfied, 
thus the relation always holds for all values D and E might take on.
}


\section{Notable Definitions from Part 1}
\subsection{Consequences}

\noindent{
    \large
    
    \textbf{Logical Consequence}: $A_1 \dots A_n$ implies B, and B is a consequence of $A_1 \dots A_n$,
    means when $A_1 \dots A_n$ are all true, then B must be true also. \\
    
    \textbf{Case}: A case be loosely interpreted a particular combination of values for variables.\\

    \textbf{Valid Argument}: An argument consist of a set of premises and a single conclusion,
    this argument is \emph{valid} if the conclusion is a \emph{logical consequence} of the premises.\\

    \textbf{Counter Example}: A counter example to an argument is a case where all premises are truth,
    but the conclusion is false.\\

    \textbf{Sound Argument}: An argument is sound if the premises are true in \emph{all} cases, 
    and the arugment is valid. An argument cannot be sound if its not already valid.\\
}

\subsection{Language}

\noindent{
    \large

    \textbf{Syntax}: Syntax consist of a basic set of symbols, and a rule set to create more 
    complex words \& sentences from symbols. Syntax is not concerned with \emph{meaning} of any symbols or sentences\\

    \textbf{Semantics}: Semantics of a language assigns meaning to a sentence in the language.\\


    \textbf{Atom, Connectives, Molecules}: An atomic sentence is the mostly basic sentence that cannot be reduced
    further, like 'sky is blue' or 'Bob is eating', atomic sentence do not have connectives. A molecular sentences is made with a number of atomic 
    sentences linked with connectives, like 'Bob is sleeping \emph{or} eating', 'Sun is bright \emph{and} hot'.
}

\subsection{Basics of Set Theory}

\noindent{
    \large

    \textbf{Set}: A set is an arbitrary, unordered \emph{collection} of unique \emph{things}, depending on context,
    duplicates are usually ignored. 2 sets are equal if they contain indentical items. For example:
    $$Food := \{apple, cookie, burger\} = \{ apple, apple, apple, burger, cookie\}  $$
    
    \textbf{Membership}($\in, \notin$): For any set it is possible to tell if an item belongs in the set. For exmaple:
    $$
    cookie \in Food, dirt \notin Food$$
    Which means that 'cookie' is in the set of Food(cookie is a member of Food), but dirt is not.\\


    \textbf{Set builder notation}: A notation used to contruct sets from definitions. For exmaple:
    $$
    L = \{n \in \mathbb{N} : n > 44\}
    $$
    Here the ':' means 'such that', so the set L is the set all natural numbers, n, such that n is larger than 44.\\

    \textbf{Union}($\cup$): The union of 2 sets is a set containing items from either sets:
    $$\{1, 3, 7\} \cup \{2, 3, 2\} = \{1, 2, 7, 3\}
    $$

    \textbf{Intersec}($\cap$): The intersection of 2 sets is a contain items that belongs to both sets:
    $$\{1, 3, 7\} \cup \{1, 2, 3, 4\} = \{1, 3\}
    $$

    \textbf{Subsets}($\subseteq $): $A \subseteq B$ if A is contained in B, that is,
    every item in A is also in B.\\Note: $A = B \iff (A \subseteq B)\wedge(B \subseteq A)$.\\

    \textbf{Proper Subset}($\subset$): A is a proper subset of B if A $\subseteq$ B, and B 
    is strictly bigger, that is, contains at least one item A does not.
}
\newpage
\subsection{Pairs and Relations}

\noindent
{
    \large

    \textbf{Ordered Pair}: Unlike sets, ordered pairs/n-tuple are ordered. So $\{a,b\} = \{b,a\}$, 
    however, $\langle a, b \rangle \not = \langle b, a \rangle$. N-tuples contains n ordered items.\\

    \textbf{Cartesian Product}: $A \times B$ is the cartesian product of A and B, which is a set containing 
    all possible ordered pairs $\langle a, b \rangle, a\in A, b \in B$. $\times$ can be applied more than 2 times. For exmaple:
    $$
    \{a, b, c\} \times \{1, 2\} = \{
        \langle a, 1 \rangle,
        \langle a, 2 \rangle,
        \langle b, 1 \rangle,
        \langle b, 2 \rangle,
        \langle c, 1 \rangle,
        \langle c, 2 \rangle\}
    $$
    $$
    D \times E \times F = \\
    \{
        \langle d_1, e, f_1 \rangle,
        \langle d_1, e, f_2 \rangle,
        \langle d_1, e, f_3 \rangle,
        \langle d_2, e, f_1 \rangle,
        \langle d_2, e, f_2 \rangle,
        \langle d_2, e, f_3 \rangle
    \}
    $$
    $$
    \text{Where }D = \{d_1, d_2\}, E = \{e\}, F = \{f_1, f_2, f_3\}
    $$\\

    \textbf{Relations}: A relation $\mathcal{R}$ on sets A and B, is a way
    to relate elements of A and B. For $a \in A, b\in B$, $a$ and $b$ 
    are in relation $\mathcal{R} \iff \langle a, b \rangle \in \mathcal{R}$, and 
    we can write $a\mathcal{R}b$.\\     Note: $\mathcal{R} \subseteq A \times B$.\\

    \textbf{Reflexivity}: A relation $\mathcal{R}$ is reflexive when $x\mathcal{R}x$ for all $x$.\\

    \textbf{Symmetry}: $\mathcal{R}$ is symmetric when $x\mathcal{R}y \iff y\mathcal{R}x$\\

    \textbf{Transitivity}: $\mathcal{R}$ is transitive when $x\mathcal{R}y, y\mathcal{R}z \implies x\mathcal{R}z$.\\

    \textbf{Equivalence}: $\mathcal{R}$ is an equivalence relation if $\mathcal{R}$ is 
    reflexive, symmetric, and transitive.\\

    \textbf{Function}: Like functions in calculus, $f: x \rightarrow y$ sends each $x$ to 1 $y$ only,
    that is, the value of $f(x)$ is not ambiguous.

    }

\newpage
\section{Classical Logics}
For logic operators and tableaux references, see Section 1.

\subsection{Turntile($\vdash $) vs Double turntile($\models$)}

\noindent
{\large


    \textbf{Turntile}: `$\vdash$' denotes \emph{syntatic} implication. 
    $A \vdash_1 B$ means with only information from $A$, it is possible to prove $B$.
    Or alternatively, it is possible to obtain $B$ from `rearranging' symbols of $A$.\\ 

    \textbf{Double turntile}: `$\models$', denotes \emph{semantic} implication, or models.
    $A \models_1 B$ means that $B$ is true whenever $A$ is true. \\

    \textbf{Notes}: A logic system is \emph{sound} if $A\vdash B \implies B$, is 
    \emph{complete} if $A \models B \implies A \vdash B$. Classical logics is sound and 
    complete so there isn't a big difference between the 2
}
\end{document}