\documentclass{article}
\usepackage{amsmath}
\usepackage{amssymb}
\usepackage{geometry}
\usepackage[linguistics]{forest}
\usepackage{microtype}
\usepackage{float}
\usepackage{makecell}
\usepackage{hyperref}
\usepackage{hypcap}
\newcommand{\true}{{\mathfrak{t}}}
\newcommand{\false}{{\mathfrak{f}}}
\geometry{legalpaper, portrait, margin=1in}
\title{Phil 120, Review Notes}
\author{Stuff}
\date{May 23, 2021}

\begin{document}
\maketitle
\newpage
\tableofcontents{}

\newpage    


\section{References}

\subsection{Basic Logic Operators}
The followings are for A * B, where '*' is an operator, A is top row, B is left column.


\begin{table}[H]
    \centering
    \large
    \begin{tabular}{c l}
        \begin{tabular}{c|c|c}
            $\wedge$ & T & F\\
            \hline
            T & T & F\\
            F & F & F\\
        \end{tabular} & {AND. Conjuction. A $\wedge$ B is true only when both A and B are true.}\\
        &\\
        \hline
        &\\
        \begin{tabular}{c|c|c}
            $\vee$ & T & F\\
            \hline
            T & T & T\\
            F & T & F
        \end{tabular} & OR. Disjunction. A $\vee$ B is true when either A or B, or Both are true. \\
        &\\
        \hline
        &\\
        \begin{tabular}{c|c|c}
            $\rightarrow$ & T & F\\
            \hline
            T & T & T\\
            F & F & T\\
        \end{tabular} & \makecell[l]{IMPLIES. If A then B. A implies B.\\
        A implies B is true when A is true and B is true, or when A is false.\\
        Note: $A \rightarrow B = \lnot A \vee B$}\\
        &\\
        \hline
        &\\
        \begin{tabular}{c|c|c}
            $\leftrightarrow$&  T & F\\
            \hline
            T & T & F\\
            F & F & T
        \end{tabular} & \makecell[l]{IFF, A if and only B. A is logically equivalent to B, two way implication.\\
        A $\leftrightarrow$ B is true exactly when the truth value of A is the same as B.\\
        Note: $A \leftrightarrow B = (A \rightarrow B) \wedge (B \rightarrow A) = (A \wedge B) \vee (\lnot A \wedge \lnot B)$
        }
    \end{tabular}
\end{table}


\subsection{Some Logic Identities}

\begin{table}[H]
    \centering
    \large
    \begin{tabular}{c l}
        $A \vee \lnot A = T$ & \makecell[l]{
            Excluded Middle, either A or not A must be true.\\
            (Not true in K3, since it could be the case that both \\ $A$ and $\lnot A$ are not true)
        }\\
        &\\
        \hline
        &\\
        $\lnot (A \wedge \lnot A)$ & \makecell[l]{
            Non-contradiction.\\ It is true that not both A and not A hold at the \\same time. 
            \\(Not true in K3)
        }\\
        &\\
        \hline
        &\\
        $A \rightarrow B, A \implies B$ & \makecell[l]{Modus ponens, to prove. \\
        If A implies and B and A is true, then B is true.}\\
        &\\
        \hline
        &\\
        $A \rightarrow B, \lnot B \implies \lnot A$ & \makecell[l]{Modus tollens, to disprove. \\
        If the conclusion is false, then the premise must be false also.}\\
        &\\
        \hline
        &\\
        $A \rightarrow A\vee B$ & \makecell[l]{Disjunctive Introduction\\
        If A is true, then A or B is also true, regardless if what B is.}\\
        &\\
        \hline

        &\\
        $A \vee B, \lnot A \implies B$ & \makecell[l]{
            Disjunctive syllogism.\\
            If at least one of A or B is true, then if one of them is \\false, the other must be true.
        }\\
        &\\
        \hline
        &\\
        $(A \rightarrow B) \iff (\lnot B \rightarrow \lnot A)$ & Contrapositive. Similar to Modus tollens.\\
        &\\
        \hline
        &\\
        $A, \lnot A \implies B$ & \makecell[l]{
            Explosion. \\
            From a false premise you can arrive at any conclusion.
        }\\
        &\\
        \hline
        &\\
        \makecell{
            $\lnot (A \vee B) \iff \lnot A \wedge \lnot B$\\
            $\lnot (A \wedge B) \iff \lnot A \vee \lnot B$
        } & De Morgan's Law: $\lnot$ distributes over $\vee, \wedge$\\
        &\\
        \hline
        &\\
        \makecell{
            $A \vee (B \wedge C) \iff (A\vee B) \wedge (A \vee C)$\\
            $A \wedge (B \vee C) \iff (A \wedge B)\vee (A \wedge C)$
        } & Distributability: $\vee$ distrubutes over $\wedge$, and vice versa.

    \end{tabular}
\end{table} 

\

\subsection{Tableaux Identities}
\begin{table}[H]
    \centering
    \Large
    \begin{tabular}{c c | c c}
        $\true \wedge$: & 
        \begin{forest}
        [$\true A \wedge B$ 
            [$\true A$ \\ $\true B$]
        ]
        \end{forest}
        & $\false \wedge$: & 
        \begin{forest}
        [$\false A \wedge B$ 
            [$\false A$]  [$\false B$]
        ]
        \end{forest}\\
        
        %%%%%%%%%%%%%%%%%%%%%%%%%%%%%%%%%%%%%%%

        \hline
        $\true \vee$: & 
        \begin{forest}
            [$\true A \vee B$ 
                [$\true A$]  [$\true B$]
            ]
        \end{forest} 
        & $\false \vee$: 
        & \begin{forest}
            [$\false A \vee B$ 
                [$\false A$ \\ $\false B$]
            ]
            \end{forest}\\
        %%%%%%%%%%%%%%%%%%%%%%%%%%%%%%%%%%%%%%%
        \hline 
        $\true \rightarrow$: &
        \begin{forest}
            [$\true A \rightarrow B$
                [$\true \lnot A$]
                [$\true B $]
            ]
        \end{forest}
        & $\false  \rightarrow $: &
        \begin{forest}
            [$\false A \rightarrow B$ 
                [$\false \lnot A$ \\ $\false B$]
            ]
            \end{forest}\\
        %%%%%%%%%%%%%%%%%%%%%%%%%%%%%%%%%%%%%%%
        \hline
        $\true \leftrightarrow$ :
        & \begin{forest}
            [ $\true A \leftrightarrow B$
                [$\true \lnot A$ [$\true \lnot B$]]
                [$\true \lnot A$ [$\true A$]]
                [$\true \lnot B$ [$\true  B$]]
                [$\true  A$ [$\true  B$]]
            ]
        \end{forest}
        & $\false \leftrightarrow$: 
        & \begin{forest}
            [$\false A \leftrightarrow B$
            [$\false \lnot A$ [$\false B$]]
            [$\false \lnot B$ [$\false A$]]
            ]
        \end{forest}\\


        \hline
        $\true \forall$ :
        & \begin{forest}
            [ $\true \forall x \ Ax $
                [$\true A\alpha$]
            ]
        \end{forest}
        & $\false \forall$: 
        & \begin{forest}
            [$\false \forall x \ Ax$
                [$\false A \eta$]
            ]
        \end{forest}\\
        \hline
        $\true \exists$ :
        & \begin{forest}
            [ $\true \exists x \ Ax $
                [$\true A\eta$]
            ]
        \end{forest}
        & $\false \exists$: 
        & \begin{forest}
            [$\false \exists x \ Ax$
                [$\false A \alpha$]
            ]
        \end{forest}\\
        \hline
        $\true =$: &
        \begin{forest}
            [{$\true a = b$ \\ $\true A(a)$}
            [
                $\true A(b)$
            ]
            ]
        \end{forest}
        & $\false =$: 
        & \begin{forest}
            [{$\true a = b$ \\ $\false A(a)$}
            [$\false A(b)$]
            ]
        \end{forest}

    \end{tabular}
\end{table}


\noindent{\large Quantifier \& Identity Note:

\begin{itemize}
    \item $\alpha$ means an arbitrary choice of variable name could used in place of $\alpha$, might be a good idea to just use $a$.
    \item $\eta$ means a new variable name that has previously unused in the current branch must be used to replace $\eta$.
    \item Use the following to deal with $\lnot$: $$ \lnot \exists x \ Ax \iff \forall x \ \lnot Ax$$ $$ \lnot \forall x \ Ax \iff \exists x \ \lnot Ax $$ $$(\false \lnot (a = b)) \iff (\true a = b)$$
    \item Identiy closure condidtions: $\false a = a, \ \true \lnot(a = a)$.
    \item Remember that $=$ is symmetric, so that $a = b \iff b = a$.
    \item Branches involving quantifiers are undecidable, so some strategies and luck is involved in closing those branches.
    \item {Quantifier order matters!: \begin{itemize}
        \item Swapping from $\exists x \forall y$ to $ \forall y \exists x$ is okay, the reserve is not, in general.
    \end{itemize}
        }
\end{itemize}
}




\noindent{\large Note: 
\begin{itemize}
    \item Use De Morgan's Law to deal with negations($\lnot$), also, $\lnot (A \rightarrow B) = (\lnot B \wedge A)$.
    \item To prove a consequence, set the premise to true, conclusion to false. This proves that there is
     no possible counter example, since the negation is never satisfied. 
     \begin{itemize}
         \item If there are atomic branches left open, then those are valid counter examples.
     \end{itemize}
    \item A branch is closed any of following pairs occurs in a branch:
    $\{(\false A, \true A), (\false A, \false \lnot A), (\true A, \true \lnot A)\}$, use a '$\times$' to 
    indicate a closed branch.
\end{itemize}
Example: Use tableaux to prove $ (E \rightarrow D) \vdash_1 ((D \wedge E) \leftrightarrow E)$\\


\begin{center}
    \begin{forest}
        [
            $\true (E \rightarrow D)$
            [$\false ((D\wedge E) \leftrightarrow E)$
                [$\true \lnot E$
                    [$\false \lnot (D \wedge E)$ 
                        [$\false E$
                            [$\false \lnot D$ 
                            [$\false \lnot E$\\$\times$]]
                        ]
                    ]
                    [$\false (D \wedge E)$ 
                        [$\false \lnot E$\\$\times $
                        ]
                    ]
                ]
                [$\true D$
                [$\false \lnot (D \wedge E)$ 
                    [$\false E$
                    [$\false \lnot D$ 
                    [$\false \lnot E$\\$\times$]]
                    ]
                ]
                [$\false (D \wedge E)$ 
                    [$\false \lnot E$
                    [$\false D$\\$\times$]
                    [$\false E$\\$\times$]
                    ]
                ]
                ]
            ]
        ]
        \end{forest}
\end{center}


Since all branches are closed, the negation of the conclusion is never satisfied, 
thus the relation always holds for all values D and E might take on.
}

\newpage

\section{Notable Definitions from Part 1}
\subsection{Consequences}

\noindent{
    \large
    
    \textbf{Logical Consequence}: $A_1 \dots A_n$ implies B, and B is a consequence of $A_1 \dots A_n$,
    means when $A_1 \dots A_n$ are all true, then B must be true also. \\
    
    \textbf{Case}: A case be loosely interpreted a particular combination of values for variables.\\

    \textbf{Valid Argument}: An argument consist of a set of premises and a single conclusion,
    this argument is \emph{valid} if the conclusion is a \emph{logical consequence} of the premises.\\

    \textbf{Counter Example}: A counter example to an argument is a case where all premises are truth,
    but the conclusion is false.\\

    \textbf{Sound Argument}: An argument is sound if the premises are true in \emph{all} cases, 
    and the arugment is valid. An argument cannot be sound if its not already valid.\\
}

\subsection{Language}

\noindent{
    \large

    \textbf{Syntax}: Syntax consist of a basic set of symbols, and a rule set to create more 
    complex words \& sentences from symbols. Syntax is not concerned with \emph{meaning} of any symbols or sentences\\

    \textbf{Semantics}: Semantics of a language assigns meaning to a sentence in the language.\\


    \textbf{Atom, Connectives, Molecules}: An atomic sentence is the mostly basic sentence that cannot be reduced
    further, like 'sky is blue' or 'Bob is eating', atomic sentence do not have connectives. A molecular sentences is made with a number of atomic 
    sentences linked with connectives, like 'Bob is sleeping \emph{or} eating', 'Sun is bright \emph{and} hot'.
}

\subsection{Basics of Set Theory}

\noindent{
    \large

    \textbf{Set}: A set is an arbitrary, unordered \emph{collection} of unique \emph{things}, depending on context,
    duplicates are usually ignored. 2 sets are equal if they contain indentical items. For example:
    $$Food := \{apple, cookie, burger\} = \{ apple, apple, apple, burger, cookie\}  $$
    
    \textbf{Membership}($\in, \notin$): For any set it is possible to tell if an item belongs in the set. For exmaple:
    $$
    cookie \in Food, dirt \notin Food$$
    Which means that 'cookie' is in the set of Food(cookie is a member of Food), but dirt is not.\\


    \textbf{Set builder notation}: A notation used to contruct sets from definitions. For exmaple:
    $$
    L = \{n \in \mathbb{N} : n > 44\}
    $$
    Here the ':' means 'such that', so the set L is the set all natural numbers, n, such that n is larger than 44.\\

    \textbf{Union}($\cup$): The union of 2 sets is a set containing items from either sets:
    $$\{1, 3, 7\} \cup \{2, 3, 2\} = \{1, 2, 7, 3\}
    $$

    \textbf{Intersect}($\cap$): The intersection of 2 sets is a set containing items that belongs to both sets:
    $$\{1, 3, 7\} \cup \{1, 2, 3, 4\} = \{1, 3\}
    $$

    \textbf{Subsets}($\subseteq $): $A \subseteq B$ if A is contained in B, that is,
    every item in A is also in B.\\Note: $A = B \iff (A \subseteq B)\wedge(B \subseteq A)$.\\

    \textbf{Proper Subset}($\subset$): A is a proper subset of B if A $\subseteq$ B, and B 
    is strictly bigger, that is, contains at least one item A does not.
}
\newpage
\subsection{Pairs and Relations}

\noindent
{
    \large

    \textbf{Ordered Pair}: Unlike sets, ordered pairs/n-tuple are ordered. So $\{a,b\} = \{b,a\}$, 
    however, $\langle a, b \rangle \not = \langle b, a \rangle$. N-tuples contains n ordered items.\\

    \textbf{Cartesian Product}: $A \times B$ is the cartesian product of A and B, which is a set containing 
    all possible ordered pairs $\langle a, b \rangle, a\in A, b \in B$. $\times$ can be applied more than 2 times. For exmaple:
    $$
    \{a, b, c\} \times \{1, 2\} = \{
        \langle a, 1 \rangle,
        \langle a, 2 \rangle,
        \langle b, 1 \rangle,
        \langle b, 2 \rangle,
        \langle c, 1 \rangle,
        \langle c, 2 \rangle\}
    $$
    $$
    D \times E \times F = \\
    \{
        \langle d_1, e, f_1 \rangle,
        \langle d_1, e, f_2 \rangle,
        \langle d_1, e, f_3 \rangle,
        \langle d_2, e, f_1 \rangle,
        \langle d_2, e, f_2 \rangle,
        \langle d_2, e, f_3 \rangle
    \}
    $$
    $$
    \text{Where }D = \{d_1, d_2\}, E = \{e\}, F = \{f_1, f_2, f_3\}
    $$\\

    \textbf{Relations}: A relation $\mathcal{R}$ on sets A and B, is a way
    to relate elements of A and B. For $a \in A, b\in B$, $a$ and $b$ 
    are in relation $\mathcal{R} \iff \langle a, b \rangle \in \mathcal{R}$, and 
    we can write $a\mathcal{R}b$.\\     Note: $\mathcal{R} \subseteq A \times B$.\\

    \textbf{Reflexivity}: A relation $\mathcal{R}$ is reflexive when $x\mathcal{R}x$ for all $x$.\\

    \textbf{Symmetry}: $\mathcal{R}$ is symmetric when $x\mathcal{R}y \iff y\mathcal{R}x$\\

    \textbf{Transitivity}: $\mathcal{R}$ is transitive when $x\mathcal{R}y, y\mathcal{R}z \implies x\mathcal{R}z$.\\

    \textbf{Equivalence}: $\mathcal{R}$ is an equivalence relation if $\mathcal{R}$ is 
    reflexive, symmetric, and transitive.\\

    \textbf{Function}: Like functions in calculus, $f: x \rightarrow y$ sends each $x$ to 1 $y$ only,
    that is, the value of $f(x)$ is not ambiguous.

    }

\newpage
\section{Classical Logics}
For logic operators and tableaux references, see Section 1.

\subsection{Turntile($\vdash $) vs Double turntile($\models$)}

\noindent
{\large
    \textbf{Turntile}: `$\vdash$' denotes \emph{syntatic} implication. 
    $A \vdash_1 B$ means with only information from $A$, it is possible to prove $B$.
    Or alternatively, it is possible to obtain $B$ from `rearranging' symbols of $A$.\\ 

    \textbf{Double turntile}: `$\models$', denotes \emph{semantic} implication, or models.
    $A \models_1 B$ means that $B$ is true whenever $A$ is true. \\

    \textbf{Notes}: A logic system is \emph{sound} if $A\vdash B \implies B$, is 
    \emph{complete} if $A \models B \implies A \vdash B$. Classical logics is sound and 
    complete so there isn't a big difference between the two symbols used.
}

\subsection{Cases}
\noindent
{
    \large

    \textbf{True/False in case}: For a particular case $c$, $c \models_1 A$ means
    `$A$ is true in case $c$', while $c \models_0 A$ means
    `$A$ is false in case $c$'.\\


    \textbf{Complete/Consistant Cases}: 
    \begin{itemize}
        \item A case is \emph{complete} if at least 1 of $c\models_1 A, c\models_0 A$ holds.
        \item A case is \emph{consistant} if at most 1 of $c\models_1 A, c\models_0 A$ holds(so not both).
        \item All cases in classical logic is both complete and consistant.
    \end{itemize}
    *Skipping things covered in Section 1 or informally covered in previous sections*
}


\subsection{Analytic Tableaux}
For example and references on tableaux method of proof, see section 1.3.\\

\noindent
{\large

\textbf{Tableaux Method}: The tableaux method provides a systematic way to determine if a statement is:
\begin{itemize}
    \item true / a theorem, if all atomic leaves of the tree are not closed.
    \item false / a contradiction, if all branches are closed.
    \item contingent, in which case the remaining unclosed branches determines the truth value of the statement.
\end{itemize}
In particular, to prove a statement, it is helpful to start with the negation/counter example instead, then try to show 
the negation is never satisfied. So when proving $A \vdash_1 B$, start with:
$$
\true A
$$
$$
\false B
$$

Then follow the reference and example in section 1.3, if all branches closes, then that means the 
condition of A been true, while B is false, is never satisfied, thus no counter example exists, therefore
$A \vdash_1 B$.\\

When there some branches left open, then that branche's nodes together forms a counter example when their truth
value is same as their sign($\true$ or $\false$). \\

\textbf{Sign of Node}: For notation in this class, each node of a tableaux is \emph{signed}($\true$ or $\false$),
which are essentially assuming the truth value of that node. \\Note, $\true$ and $\false$ signed 
nodes splits differently, and children nodes have same sgin as their parent. Also:
$$
\true A = \false \lnot A, \false A = \true \lnot A
$$
}

Skipping chapter 7, doesn't look important.
\newpage


\section{First Order Logic}
\subsection{Predicates}

\noindent
{\large

\textbf{Unary Predicate}: A single variabled \emph{well formed formula}(or a boolean function) that 
evaluates to either true or false depending on the variable. In this class, a predicate looks like $Pa$ 
where $a$ is the variable, P is the predicate.\\

\textbf{`Denotes'}: Suppose \emph{Object} has name of \emph{`Name'}, then \emph{`Name'} denotes 
\emph{Object}, and \emph{Object} is the denotation of \emph{`Name'}\\

\textbf{Extension \& AntiExtension}: The \emph{Extension} is the set of objects(denotations) that makes 
the predicate true. The anti-extension makes the predicate false. For example, suppose $Px$ is true 
if $x$ is red:
$$
+(P) = \{apple, cherry, stop \ sign, \dotsb\}, \ -(P) = \{sky, orange, Phil \ 120 \ midterm, \dotsb\}
$$
\textbf{Domain of Discourse}: Domain of Discourse, $D$, for a predicate is a set of objects applicable to the 
predicate. So if the predicate is `x is the first day of the week', the $D$ might be the set of all days in a week.
Note: $+(P)\cup-(P) = D$, and $+(P) \cap -(P) = \emptyset$, in order for the predicate to behave classically.\\

\textbf{Denotation Function}($\delta$): The denotation function, $\delta(a)$, gets the denotation from 
the domain of Discourse which $a$ denotes. For a predicate $P$, $Pa$ is true iff $\delta(a) \in +(P)$\\

\textbf{Interpretation}: A interpretation(\emph{case}) contains a domain of discourse and
a denotation function($\langle D, \delta \rangle$), which is a more formal definition of 
a \emph{case}, which previous parts have been mentioning.\\

Interpretations can be used to satisfy statements, or to serve as counter examples, like in A2 Q2:\\

a)
Find an interpretation which satisfy $\varGamma = \{(\lnot Ha \vee Hb),(\lnot Ha \rightarrow \lnot Hb),
(\lnot Hc \vee Hd)\}$
\begin{flalign*}
    & \text{Define } \mathcal{I} =\langle  {D,\delta}\rangle \\
    & \text{Where } D = \{o_1, o_2, o_3, o_4\}\\ 
    & \text{and } \delta(a) = o_1, \delta(b) = o_2, \delta(c) = o_3, \delta(d) = o_4\\
    & \text{such that } +(H) = \{o_1, o_2, o_3, o_4\}, -(H) = \{\} 
\end{flalign*}
Now, $Ha, Hb, Hc, Hd = T \\\implies \mathcal{I} \models_1 (\lnot Ha \vee Hb), \ 
\mathcal{I} \models_1 (\lnot Ha \rightarrow \lnot Hb),\ 
\mathcal{I} \models_1 (\lnot Hc \vee Hd)\ \\
\implies \mathcal{I} \models_1 \varGamma
$
\\\\
b)
Find a counter exmaple to $\{\lnot Ga, (Ga \rightarrow Pb), (\lnot Pb \rightarrow Ga)\}
\models_1 (Ga \vee \lnot Pb)$
\begin{flalign*}
    & \text{Define } \mathcal{I} =\langle  {D,\delta}\rangle \\
    & \text{Where } D = \{o_1, o_2\}\\ 
    & \text{and } \delta(a) = o_1, \delta(b) = o_2\\
    & \text{such that } +(G) = \{\}, -(G) = \{o_1\}, +(P) = \{o_2\}, -(P) = \{\}
\end{flalign*}
So that in this interpreation Ga is false, while Gb is true\\
$\implies \lnot Ga = T, \  (Ga \rightarrow Pb) = T, \  (\lnot Pb \rightarrow Ga) = T$.\\
However $(Ga \vee \lnot Pb) = F$, so in this interpretation, the premise is true but the conclusion
is false, which makes this a counter example.

}

\newpage

\subsection{Quantifiers}

\noindent
{
    \large 

\textbf{Motivation}: To decribe statements like ``Every apple is sweet'', ``Some cities rains all the time.'', 
which contains quantifiers.\\

\textbf{Universal and Existential}($\forall, \exists$): 
\begin{itemize}
    \item {`$\forall$' translates into `for all', `every', `any', is used to specify any and all of the items in a set:
        $$
        \forall n \in \mathbb{N}, 2n > n. \text{ Meaning for all natural number n, 2n is larger than n.}
        $$
    }
    \item {`$\exists$' translates in to `exists', `at least one', `some', is used to specify at least one of an item in a set:
    $$
    \exists z \in \mathbb{Z}: z^2 = z. \text{ There exist an integer $z$, such that $z^2 = z$.}
    $$
    }
    \item Quantifiers are tied to variables within its scope. In $P(z) := \exists x (Px\vee Pz) \wedge (\forall y\  \lnot Py \rightarrow Px)$, 
    $\exists x$ applies to the whole formula, but $\forall y$ only applies to the inner bracket. Here $z$ is a \emph{free} variable, 
    since it is not bound by a quantifier.
    \item Order matters for quantifiers, for example:$$
        \exists x \forall y \in \mathbb{R}, x < y \text{ is false, since there is no smallest element in $\mathbb{R}$}
    $$
    $$
        \forall y \exists x \in \mathbb{R}, x < y \text{ however holds, since for all $y$, there is a $x$ that is smaller.}
    $$
    \item To prove or find counter example: \begin{itemize}
        \item To prove $\forall x, Px$, it is necessary to show $Px$ holds for any arbitrary $x$.
        \item To disprove/find counter example for $\forall x, Px$, it is sufficient to find a single $x$ such that $Px$ does not hold.  
        \item To prove $\exists x, Px$, it is sufficient to show there is at least 1 $x$ such that $Px$ holds.
        \item To disprove $\exists x, Px$ it is necessary to show $Px$ is not satisfied for all $x$.    
    \end{itemize}
\end{itemize}



\textbf{Name vs Variable}: Name is refering to a static named reference of a object, while 
a variable is an unknown. For example : ``Bob is short, and for any person, $p$, p is friends with Bob. $x$ is even
shorter than Bob''. Here Bob is \emph{named}, while $p$ is a variable bound by quantifier, and $x$ is a free variable since
it is not bound by a quantifier.\\

\textbf{Valuation Function}: For a particular interpretation, a valuation function may be added $\rightarrow \langle D, \delta, v \rangle$.
$\delta(a)$ is used to get the object in $D$ that is used in place of names, while $v(x)$ is used to 
get object to be used in place of \emph{variable}. $v$ is similar to $\delta$, but for variable.\\

\textbf{Prof's Definitions for Quantifier}: First, for any particular valuation function $v$,
define $v'$ such that $v' \mathtt{\sim}_x v$, $v'$ returns the same value as $v$, except for $v'(x)$. Now, 
universal and Existential quantifier may be define like so:
\begin{itemize}
    \item {$w \models^v_1 \forall x Gx \iff \text{ for any } v' \mathtt{\sim}_x v, w \models^{v'}_1 Gx$.
    \\So no matter which object in $D$ is put in place of $x$, the statement holds still.
    } 
    \item {$w \models^v_1 \exists y Gy \iff \text{ there is at least 1 } v' \mathtt{\sim}_y v, w \models^{v'}_1 Gy$.
    \\So at least 1 object exist in $D$, such that when put in place of y, the statement holds.
    } 
\end{itemize}

\textbf{Predicate with Multiple Free Variable}: A predicate may have more than 1 variable, for example $Gab$, 
for the predicate of `$a < b$. Predicate with multiple variable could be written like `$P^n$', where 
$n$ is the number of variable. For a predicate of $n$ variable, then the extension/anti-extension should contain 
\emph{n-tuples} since the predicate needs n variables.
}

\newpage
\subsection{Identity}

\noindent{
    \large

\textbf{Motivation}: To determine if 2 objects are `Equal', that they behave identically, or have 
the same properties.\\

\textbf{Extention and AntiExtention}: 
$$
+(=) \text{ is } \{\langle o, o \rangle : o \in D\} \text{ (the identity relation, or $Id_D$)}
$$
$$
-(=) \text{ is } \{\langle o_1, o_2 \rangle : o_1, o_2 \in D, \space o_1 \not = o_2\}
$$

Such that everything is identical to itself, and nothing else. Recall that a equivalence relation
is reflexive($x = x$), symmetric($x=y \iff y = x$), and transitive($x=y \wedge y = z \implies x = z$).
\\

\textbf{First Order Leibniz Law}: if $a=b$, then $a$ is interchangable with $b$:
$$
Pa, a = b \models_1 Pb
$$

\textbf{Properties of $\not =$}:
 
\begin{itemize}
    \item $(a\not= b \iff \lnot (a = b))$
    \item $\forall x, x \not = x$ is always an contradiction, $\not = $ is not reflexive
    \item $\forall x, y \quad x \not = y \iff y \not = x$, $\not = $ is symmetric
    \item In general $\exists x, y, z : x \not = y \wedge y \not = z \not \implies x \not = z$, $\not =$ is not transitive
    \item $Pa, a\not = b \not \models_1 Pb$
\end{itemize}
For tableaux formulas involving quantifier and equality, see the reference section.
}

\newpage
\section{Other Logcial systems}

\large
This section examines logical systems with 3(Neither) or 4(Neither and Both) logical values. For example, 
``it is raining right now.'' is either true or false, but ``it will rain tormorrow'' is indeterminant, since
it's not possible to know for sure if it will rain tormorrow.
\\

\subsection{CL, K3, LP, FDE}
\large
Options for alternative logic systems:\\

\begin{tabular}{c | c| l}
    \hline
    Consistant & Complete & Logical Theory\\
    \hline
    Yes & Yes & CL Classical Logics\\
    
    Yes & No & K3 Strong Kleene\\
    No & Yes & LP Logics of Paradox \\
    No & No & FDE First-Degree Entailment\\
    \hline
    
\end{tabular}\\  \\ \\


\noindent\textbf{Paracomplete \& Paraconsistent}: 
\begin{itemize}
    \large
    \item A logical theroy is \emph{paracomplete} 
    if it is not complete, that is for some statement A, $\exists c\ :\  c\not \models_1 A \wedge c \not \models_0 A$,
    so some statement in some cases cases can be both not true, and not false at the same time.
    \item A logical theory is \emph{paraconsistent} if it is not consistent, that is, for some statement A, 
    $\exists c \ :\ c \models_1 A \wedge c \models_0 A$, so that under some cases, some statements can be 
    both true and false at once.
    \item Note: K3 is \emph{Paracomplete}, LP is \emph{Paraconsistent}, FDE is both(so neither and both true and false cases exists).
    
\end{itemize}
\bigskip

\noindent \textbf{Logical Consequences}: 
Let $\vdash_{Cl}\quad \vdash_{K3} \quad \vdash_{LP}\quad \vdash_{FDE}$ denote consequence relation in each logical system. 
Where $w \vdash_T A$ means that under the logical theory $T$, there no case where $w$ is satisfied, but $A$ isn't; or in other words,
no counter examples exists.
\begin{itemize}
    \item {$w\vdash_{K3}A \implies w\vdash_{CL} A$. This holds because logical consequence really means
     that no counter example exist, and also that classical cases are subset of Incomplete cases. So no 
     incomplete counter examples exists $\implies$ no classical counter examples exists. \\
     (This shows that a language can be incomplete(Like most natural languages) yet still have cases that is entirely precies and classical.) } 
     \item {$w\vdash_{LP}A \implies w\vdash_{CL} A$. This holds similar to reason above. (A language 
     can be inconsistent/ have paradox yet still have a subset of cases that behaves classically.
     )}
     \item Similarly, $w\vdash_{FDE}A \implies w\vdash_{CL} A, \ w\vdash_{LP}A, \ w\vdash_{K3}A$

\end{itemize}

\bigskip
\noindent \textbf{Semantic Values}: Each theory have a set of \emph{semantic values}, which are results of 
interpreting a statement, in classical theory, a statement is either true or false, so $\mathcal{V}_{CL} = \{T, F\}$. The 
semantic values for other thories are as follows:\\

\begin{tabular}{c|l}
    Theory & Semantic Value\\
    \hline
    CL & $\{T, F\}$\\
    K3 & $\{T, F, N\}$\\
    LP & $\{T, F, B\}$\\
    FDE & $\{T, F, B, N\}$\\

\end{tabular}\\\\
Where $N$ is semantically nethier($c \not \models_1 A \text{ and } c \not \models_0 A$), $B$ is both true and false($c  \models_1 A \text{ and } c  \models_0 A$).\\
Logical conjunctions holds the same way as in classical theory, for exmaple: \\$c \models_1 A\vee B \text{ iff } c\models_1 A \text{ or } c\models_1 B $

\newpage
\subsection{Logic Identities in K3 and LP}
When moving from classical logics to a paraconsistent or paracomplete theory, some logical identities are sacrificed.

\noindent \textbf{K3}: In K3, Excluded Middle and Non-contradiction are both lost compared to classical logics. See section 1.2.\\

\noindent \textbf{LP}: In LP, its notable that all classical true statements are also truth in LP. Also,
\emph{modus ponens, modus tollens}, Disjunctive Syllogism, and Explosion are all lost when moving from classical to LP.
\end{document}